%% The first command in your LaTeX source must be the \documentclass command.
%%
%% Options:
%% twocolumn : Two column layout.
%% hf: enable header and footer.
\documentclass[
% twocolumn,
% hf,
]{ceurart}

%%
%% One can fix some overfulls
\sloppy

%%
%% Minted listings support 
%% Need pygment <http://pygments.org/> <http://pypi.python.org/pypi/Pygments>
\usepackage{listings}
%% auto break lines
\lstset{breaklines=true}

\usepackage{rotating}
\usepackage{hyperref}
\usepackage{longtable}

%%
%% end of the preamble, start of the body of the document source.
\begin{document}

%%
%% Rights management information.
%% CC-BY is default license.
\copyrightyear{2025}
\copyrightclause{Copyright for this paper by its authors.
  Use permitted under Creative Commons License Attribution 4.0
  International (CC BY 4.0).}

%%
%% This command is for the conference information
\conference{Workshop on Computational Drama Analysis, 3 September 2025, Berlin, DE}

%%
%% The "title" command
\title{Towards a Computational Approach to News-Bringers in Ancient Greek Tragedy}

%%
%% The "author" command and its associated commands are used to define
%% the authors and their affiliations.
\author[1]{Charles Pletcher}[%
orcid=0000-0003-2734-5511,
email=charles.pletcher@tufts.edu,
url=https://github.com/pletcher,
]
\cormark[1]
\address[1]{Tufts University, Department of Classical Studies, 10 Upper Campus Road, Medford, Massachusetts 02155, United States}

%% Footnotes
\cortext[1]{Corresponding author.}

%%
%% The abstract is a short summary of the work to be presented in the
%% article.
\begin{abstract}
This paper argues that messenger speeches (\textit{Botenberichte}) in the ancient tragedies of Aeschylus, Sophocles, and Euripides have not received the computational attention that they deserve. I first draw attention to this gap by surveying traditional literary approaches to the genre, showing how the definition of ``messenger speech'' is already problematic from a qualitative perspective. Next, I present a series of initial inquiries into quantitative and computational approaches to messenger speeches, beginning with a stylometric comparison to Homeric epic and proceeding through topic modeling and, using the editions prepared by DraCor, network analysis. In conclusion, I offer a brief discussion of these preliminary findings and potential avenues for further inquiry.
\end{abstract}

%%
%% Keywords. The author(s) should pick words that accurately describe
%% the work being presented. Separate the keywords with commas.
\begin{keywords}
  tragedy \sep
  messenger speeches \sep
  Homer \sep
  topic modeling \sep
  network theory
\end{keywords}

%%
%% This command processes the author and affiliation and title
%% information and builds the first part of the formatted document.
\maketitle

\section{Introduction}

The usual history of (so-called) ``messenger speeches'' in drama begins with Homer: the ancient bard's narrative \textit{energeia}, to borrow a term from Pseudo-Longinus, is said to animate the off-stage events that news-bringers narrate. Irene J. F. de Jong has perhaps most famously applied a narratological lens to both Homer and tragic messengers, followed a few years later by Barbara Goward and James Barrett with their own seminal works on messenger speeches in 2002 \cite{deJong2011, deJong1991, Goward1999, Barrett2002}. Although de Jong championed the view that messengers are not featureless conduits for the dramatist's voice but are instead focalizers with bias and selective memory, Barrett then qualified this view by noting that messengers' narrative effectiveness is inversely proportional to their stage presence: ``The messenger's own body,'' he writes, ``is implicitly—and explicitly, on occasion—a hindrance, inasmuch as his success as a messenger depends upon his acquiring the invulnerability granted by freedom from bodily constraints'' \cite[100]{Barrett2002}. More recently, Margaret Dickin showed how the roles of messengers in tragedy increased over time and became ``vehicles for star-power,'' while others, like Felix Budelmann and Evert van Emde Boas, explored how messengers direct the audience's affective energies \cite{Dickin2009, Budelmann.vanEmdeBoas2020}. Florence Yoon's interventions have further disrupted the ``category'' of messengers by revealing the mistakes of painting with too broad a brush: \textit{angeloi}, messengers proper, and \textit{kerykes}, ``heralds,'' perform different functions in tragedy \cite{Yoon2022}. We are clearly on well-traveled ground.

Despite the steady stream of innovative approaches to messenger speeches in tragedy, there have been few attempts to address the various classes of news-bringers from a computational angle. Where such examinations have occurred, they have, to my knowledge---which admittedly might be lacking, please feel free to correct me here!---taken place in the context of broader computational studies of drama, such as the work of Julia Jennifer Beine, Frank Fischer, and Viktor Illmer presented at the DH Conference in 2024; the examination of the chorus by María Teresa Santa María Fernández and Monika Dabrowska; or the early work on social network theory and Greek tragedy by Jeff Rydberg-Cox \cite{Beine.etal2024, Fernandez.Dabrowska2023, Rydberg-Cox2011}.

In this paper, I attempt to start some discussions about the quantifiable natures of news-bringers and their speeches in ancient tragedy. Everything that I write here should be understood as experimental and preliminary---there is a surprising amount of ground to cover. At the same time, I hope that by exploring the ways that we can measure a character's ``messenger-type-ness,'' or the extent to which they perform a ``messenger function'' (to borrow a term from C. W. Marshall \cite{Marshall2006}), we can shed new light on this often-misunderstood class of dramatic actors and thus better understand how they reflect their culture's relationship to secondhand information.

\section{Epic Style?}

In some ways, what prompted this paper was a casual investigation of the \textit{Stylo} package \cite{Eder.etal2016} by way of answering the question, ``Do tragic messengers really speak with epic diction?'' The affirmative answer to this question is usually assumed, but even preliminary results raise important questions.

\begin{figure}
  \centering
  \includegraphics[width=\linewidth]{tragedy-homer-pca}
  \caption{Principal component analysis of stylometric features from individual speakers in tragedy (various colors) and Homeric epic (yellow). This analysis did not control for metrical differences, as it analyzed running 4-grams.}
\end{figure}

Although the dimensionality reduction in the analysis that produced the above chart did not account for metrical differences---Homeric epic is in dactylic hexameters, while messengers \textit{usually} speak in iambic trimeters---the strong clustering effect exhibited by meter helps to highlight some of the more interesting outliers. For one, the Tutor from Sophocles' \textit{Electra} speaks in a style that gravitates towards the Homeric. This result makes sense, however, when we consider that much of his vocabulary---and indeed, much of the scene that he describes---comes straight from the chariot race in \textit{Iliad} 23, even if he delivers the speech in iambs rather than dactyls.

More surprisingly, the Phrygian from Euripides' \textit{Orestes}---a late tragedy, first performed in 408 BCE---also exhibits a style that leans Homeric, at least compared to other speakers in tragedy. This result is unusual: unlike most of the other news-bringers, the Phrygian \textit{sings} his report in lyric meters. We might have thus expected principal component analysis to reveal him as an outlier in the \textit{other} direction. Perhaps, however, the variations in his lyric meters produced greater overlap with Homeric style than the typical iambs of other news-bringers.

Not to belabor the discussion of a quick and limited experiment, but it bears noting that the decision not to control for meter came from a consideration of the costs of doing so. When comparing tragedy to epic, we are working with surprisingly little data---a little under 250,000 words in tragedy, and about 350,000 words in the Homeric epics (the \textit{Iliad} and \textit{Odyssey}). Approaches that work well for millions of words fail to produce meaningful data on these smaller corpora, and we must consequently take care when employing any lossy data cleaning methods like lemmatization that might otherwise help us control for features like meter.

Further, if, as sometimes seems to be the case, the argument in favor of positively comparing messenger-type speeches to Homeric epic means that the audience was expected to \textit{hear} Homeric speech during a messenger's report, then meter is an indespensible component of the performance, and something to which any analysis should attend.

Although the tight clustering of tragedy and epic is apparent, a closer inspection does show that many messengers---\textit{angeloi}, \textit{kerykes}, as well as named characters like those mentioned above---tend towards the Homeric side of tragedy, even when they speak in tragic trimeters and not epic hexameters. This stylometric analysis thus provides a useful starting point for further inquiry into questions of \textit{tragic} style specifically. As Paschalis Agapitos and Andreas van Cranenberg have shown with Seneca's \textit{Octavia} and \textit{Hercules Oetaeus}, and as Nikos Manousakis and Efstathios Stamatatos have shown with Euripides' \textit{Rhesus}, stylometry can offer vital insights into cases of contested authorship \cite{Agapitos.vanCranenburgh2024, Manousakis.Stamatatos2018}. Stylometry, in this case, appears to confirm our prior intuitions while nevertheless prompting additional questions about messenger-type speeches in tragedy: what makes them distinct from other kinds of tragic speech (or song, for that matter)?

A literary question lurks here too: must we insist that tragedy answer to Homer? I have no ready answer, and I have been known to call tragedy (affectionately) ``Homer fan-fiction.'' But how does tragedy respond to itself, and does it use its subgenres, like odes and messenger-type speeches, to encode those responses?

I turn my attention to these questions for the remainder of this paper.

\section{What do news-bringers talk about?}

In this section, I discuss the use of topic modeling through Latent Dirchlet Allocation (LDA) and Non-Negative Matrix Factorization (NMF) to extract ``topics''---groups of frequently coöccurring terms---from the tragic and epic corpora, as well as from the smaller corpora of speeches in Homer and of messenger-type speeches within tragedy.

For this analysis, I identified messenger-type speeches by manually adding to the list that Barrett compiled in an appendix to his 2002 monograph \cite[223--224]{Barrett2002}. For speeches in Homer, I relied on the DICES API \cite{Forstall.etal2022}. The editions used are the latest available from the Perseus Project \cite{Crane.etal2025}. After extracting the text for each speaker from the TEI XML, the documents were sentencized, tokenized, and lemmatized using the pre-trained greCy\footnote{\url{https://spacy.io/universe/project/grecy}} models before finally being written in the CoNLL-U\footnote{\url{https://universaldependencies.org/format.html}} format. Because greCy occasionally misses stop-words, an additional stop-word filter was applied via a manually curated list before running the following analyses.

These results are preliminary, and the topics are numbered rather than descriptively labeled, so I will refer to them by their (sub-)corpus and the number assigned to them in the plot, e.g., ``Tragedy 1,'' ``Epic 5,'' etc.

The tragedy corpus and the messenger sub-corpus contained too few tokens for meaningful results from LDA, which uses TF--IDF vectors, so that terms that appear in fewer ``documents''---individual speakers, for these analyses---have proportionally greater weights. This approach already poses some challenges for small corpora, which are exacerbated by the prevalence of hapaxes in this data set. For these reasons, I will mainly focus on the results of the NMF analyses, which are plotted below.


%% Note to editors: Apologies that these plots are impossible to read. I've been fighting with
%% LaTeX for a bit to try to make them legible (and to appear in the right spot),
%% but would appreciate any advice that you have on that front.

\begin{figure}[h]
  \includegraphics[width=\linewidth,scale=0.5]{homeric-topics_NMF}
  \caption{NMF--based topic modeling of the corpus of speeches in Homer.}
\end{figure}

\begin{figure}[h]
  \includegraphics[width=\linewidth,scale=0.5]{tragic-topics_NMF}
  \caption{NMF--based topic modeling of tragedy.}
\end{figure}

\begin{figure}[h]
  \includegraphics[width=\linewidth,scale=0.5]{messenger-topics_NMF}
  \caption{NMF--based topic modeling of messenger-type speeches in tragedy.}
\end{figure}

As these topics show---and as we should expect---Athenian tragedy and Homeric epic demonstrate a significant amount of lexical overlap. At the same time, notable trends emerge, and the extracted topics by and large make sense: Homer 1, for instance, includes lemmata like μάχη (``war''), ἀργεῖος (``Argive''), θυμός (``spirit''), and μάχομαι (the verb ``fight''), representing the martial tenor of the \textit{Iliad}. Tragedy 3 includes ἀγαμέμνων (``Agamemnon''), βάρβαρος (``barbarian''---apt given the frequent accusation of Agamemnon acting in an un-Greek-like manner), γάμος (``wedding'' or ``marriage''), alongside two terms for a household, οἶκος and δόμος, all terms that cause frequent conflict in various plays.

Inspired by the work of Thomas Koentges in measuring the ``philosophical-ness'' of the Greek corpus, the top 10 topics from Homer and tragedy were measured proportionally for messenger speeches and, as a control, for tragedy as a whole \cite{Koentges2020}.

\begin{figure}[h]
  \includegraphics[width=\linewidth]{topics-by-messenger}
  \caption{Proportion of each of the top 10 topics from Homer (shades of blue) and tragedy (shades of red) for each messenger in the tragic corpus.}
\end{figure}

\begin{figure}[h]
  \includegraphics[width=\linewidth]{topics-by-tragedy}
  \caption{Proportion of each of the top 10 topics from Homer (shades of blue) and tragedy (shades of red) for each tragedy.}
\end{figure}

Although I do not have space here to discuss these results in detail, we can see at a glance that messengers speak no more ``Homerically'' than other characters in tragedy, at least in terms of vocabulary. It seems that we will need to find support elsewhere when it comes to the Homeric influence on tragic news-bringers.

Other trends are visible as well: named characters who deliver messenger-type speeches show a slight preference for tragic lexis in thos speeches, compared to anonymous news-bringers with names like ``Messenger,'' ``Shepherd,'' and ``Nurse.'' And in tragedy as a whole, the proportion of Homeric vocabulary remains remarkably consistent over the seven decades measured here, even across plays that do not touch directly on the stories contained in the epics (such as Sophocles' Theban plays or Euripides' \textit{Medea}).

\section{Audiences}

I want to turn now to a discussion of messengers' audiences in each tragedy. The DraCor documentation includes a helpful tutorial\footnote{\url{https://dracor-org.github.io/dracor-notebooks/catch-a-protagonist-in-dracor/catch-a-protagonist-in-dracor.html}} describing ways of ranking character centrality. Unsurprisingly, when applied to the tragedies in DraCor, choruses and named characters dominate the centrality rankings: the first character who can even be said to deliver a messenger-type speech is Clytaemestra in Aeschylus' \textit{Agamemnon}, who by centrality ranking comes in 45th. Talthybius in Euripides' \textit{Trojan Women} is the highest-ranking named herald at 73rd, and the first messenger-proper to appear is the Ἄγγελος (``Messenger'') of Aeschylus' \textit{Persians} at 79th. The next \textit{angelos} to appear is from Euripides' \textit{Helen}, in 96th place. The rest of the \textit{angeloi} appear in the bottom half of the ranking, making the messenger of \textit{Persians} something of an outlier. (See the \hyperref[sec:appendix]{Appendix} for a full ranked list of characters in DraCor.)

The outlier status of the messenger in \textit{Persians} is not surprising, as he speaks at greater length than any other single messenger in tragedy. To put this in perspective, nearly every named character in Sophocles' \textit{Trachiniae} delivers a messenger-type speech at some point, and the play consists of nearly twenty-five percent messenger-type speeches (as Jebb surmised over a century ago) \cite{Jebb1902}. \textit{Persians} consists of nearly seventeen percent messenger-type speech, every line of which is delivered by this single messenger. These plays contain far and away the greatest concentrations of messenger-type speeches, which on average only account for 8.5\% of a given tragedy's lines.

In some ways, this is confirmation of Trilcke et al.'s observations on ``small worlds'' in more modern plays \cite[7--33]{Andresen.Reiter2024}. A central character---usually the chorus in ancient tragedy---connects clusters of other characters in each play's network, although I am not sure that in the case of the ancient plays we can say that the chorus ``[carries] the plot'' \cite[22]{Andresen.Reiter2024}. There is also the problem of dramatic separation, discussed in greater detail below: the chorus in Greek tragedy typically occupied a distinct part of the performance space, the ὀρχήστρα or ``dancing space,'' while the action of the other actors was concentrated around the σκηνή, a wooden structure that served as a play's background and main scenery.

In the same volume as Trilcke et al.'s paper, Rebecca Hicke and David Mimno provide another way of approaching the problems of influence in drama \cite[87--105]{Andresen.Reiter2024}. By charting the percentage of words spoken by women (as a group or as specific characters) in each scene of Shakespeare's comedies, they add a chronological element that is difficult to capture in network analyses. It might be instructive, however, to examine network ``snapshots'' in ancient tragedy, comparing the graphs at the beginning and end of each episode to determine what changes were introduced and by whom during a given exchange. This approach might also provide a solution to the chorus problem, as their odes take place separately from the main episodes that drive the plot.

\section{Case Study in Movable Networks: Euripides's \textit{Andromache}}

Wanting to take a provisional stab at modeling a network over time, I have drafted a network visualization using Mike Bostock's Observable Framework and D3.js \cite{Observable2025, Bostock2025}. The experiment is visible here: \url{https://observablehq.com/@pletcher/networks-in-euripides-andromache}. For this initial pass, I encoded the major actions of Euripides' \textit{Andromache}---that is, I noted where the text indicated an entrance or an exit, as well as where characters engaged in dialogue. Capturing dialogue is particularly important for correctly modeling a network in Athenian tragedy, largely because of the chorus. Aristotle famously criticized Euripides for not integrating the chorus into his drama as well as Sophocles and Aeschylus (\textit{Poetics 1456a 25--32}), meaning that we cannot assume that just because the chorus is ``on stage''---itself a problematic descriptor!---that they actively participate in the network of the drama. As I hope to show, recording dialogue as moments when speakers actively speak to each other, rather than moments when they stand next to each other onstage, helps reveal the intricate and shifting influences that are active in tragedy.

Oliver Taplin's pioneering work on Aeschylean stagecraft sets the scene (as it were) for this experiment. His early description of the challenges of defining ``entry'' and ``exit'' is worth quoting at length:
``At first sight it might seem obvious that by 'entry' we mean the movement which brings an actor in to the field of vision of the audience, and by 'exit' we mean the movement which takes him out of it. This is certainly what we mean by the words; but their full range is more complicated. Apart from the small but valid complaints that not all of the actor comes into view at the same moment (apparently the side-entrances at Athens sloped uphill) and that he would not enter or leave the field of vision of all the audience simultaneously (especially not in a theatre of Greek shape), there is the more substantial point that we cannot possibly know at what moment the actor crossed the line between out of sight and in sight of the audience. \ldots And so in the study of entrances and exits the stage movement which is of interest is not the momentary movement in and out of view, but the prolonged movement across the centre of attention, and back again''\cite[7]{Taplin1977}.

Further complicating matters, the chorus did not enter and exit like the other actors, nor did it do so in the same place. Rather, it danced into the \textit{orchēstra} during its \textit{parodos} (``entrance song'') stayed in the \textit{orchēstra} during its \textit{stasima} (``stationary'' songs)---and left while or shortly after it sang the \textit{exodos} (``exit song''). Its continued presence during the bulk of the tragedy thus complicates ``traditional'' approaches to network analysis, even before we consider that its ``presence'' is dramatically attenuated by its literal separation from the rest of the actors, who performed in front of the \textit{skēnē}.\footnote{See, e.g., Helene Foley's ``Choral Identity in Greek Tragedy'' for a more in-depth discussion and additional bibliography on the chorus \cite{Foley2003}.} These problems become especially apparent in a drama like \textit{Andromache}, where the chorus alternates between close engagement with the dilemmas onstage and aloof (and, in this case, frankly misogynistic) reflection on feminine jealousy.

As the following figure shows, redrawing the dramatic network at each ``event'' helps to visualize when characters coexists in the performance space without addressing each other directly, as during the \textit{parodos}.

\begin{figure}[h]
  \includegraphics[width=\linewidth]{unconnected-network}
  \caption{The Chorus and Andromache are each in their respective performance spaces during the \textit{parodos}, but they are not yet talking to each other.}
\end{figure}

By contrast, when Peleus enters, the characters engage in a heated \textit{agōn} that involves every speaking character onstage and in the \textit{orchēstra}---seemingly violating the three-actor rule with the singing Child, but I leave that discussion for another day.

\begin{figure}[h]
  \includegraphics[width=\linewidth]{connected-network}
  \caption{The play's central \textit{agōn} involves every speaking character onstage at some point.}
\end{figure}

Even a quick glance at these two plots reveals a striking difference between reading a dramatic network as the static result of conversations at the end of the drama versus modeling the network as an evolving group of ephemeral events. I intend for further investigation to measure the shifts in centrality and count-based measures of influence, and I welcome discussion about how to quantify this influence over time.

I want to note, too, that \textit{Andromache} presents another, less-frequently encountered problem of mute actors at the end of the play. Scholars have long wondered whether Andromache and the Child re-enter for the final episode and \textit{exodos}, even though they do not speak, as the \textit{dea ex machina} Thetis seems to address them directly \cite{Golder1983}. For this experiment, I have not attempted to model mute actors---there are also several slaves and attendants that appear earlier in the play with Menelaus, for example---but I offer this conundrum as a potential jumping-off point for further discussion.

\section{Conclusion(s)}

By way of conclusion, I want to stress the preliminary and experimental nature of this talk. Much work remains to be done, especially in terms of addressing the challenges of meaningful network analysis within the constraints of ancient tragedy. Nevertheless, a few key themes have emerged in this brief discussion.

First, the tension between Homer and the tragedians provides a productive background for continuing investigation and discussion. The results presented here demonstrate that messenger-type speeches in tragedy lack discernible Homeric lexis, which helps us view these ancient performances in a new light. The embodied and stage-directed actions delivered alongside the words of a messenger-type speech belong to the ephemera of staged performance---we cannot recover them. Nevertheless, and contrary to much earlier work, by emphasizing the need to understand gesture and affect in addition to vocabulary and diction, we position messenger-type speeches alongside other dramatic performances, rather than an epic appendage that dramatists have maintained merely for convenience. Moreover, this examination of tragic lexis should also guide further discussion of tragic diction and performance theory, helping to make sense, for example, of the enigmatic roles that props and costumes played alongside the actors who employed them by observing how speech patterns change alongside movement. Taplin has famously tackled many of these questions from a literary and performance studies angle, and we would profitably follow his lead through our own computational engagements with the material \cite{Taplin1977, Taplin2003}.

Second, the preliminary topic modeling results presented here ought to be followed by a broader investigation involving more of the Greek corpus, similar to the work carried out by Koentges \cite{Koentges2020}. Although constraining the models to the vocabulary available from lemmatized editions of Homer and tragedy helps to narrow the focus and reveal useful groupings, like ``Odysseus'' and ``ship,'' an analysis that includes larger swaths of the ancient corpus could profitably reveal connections to history and oratory (for starters) that could not be covered in this study.

Relatedly, I have left questions of Aristophanic comedy to the side. However, given the comedian's penchant for quoting tragedy---sometimes at length, as he does with Euripides' \textit{Telephus}, \textit{Helen}, and \textit{Andromeda} in \textit{Thesmophoriazusae}---a more thorough examination of tragic-comic intertexts could help to contextualize the performance culture of fifth-century Athens to an even greater degree.

Finally, the question of tragic networks remains. In addition to the scene co-occurrences analyzed here, I plan to investigate other methods that can better account for tragedy's idiosyncracies. For example, much tragic ``action'' occurs off-stage, and it should be possible to study the networks that emerge from narrated interactions as well. One thinks, for example, of J. M. Mossman's seminal article ``Waiting for Neoptolemus,'' which discusses the strange non-presence of Neoptolemus in Euripides' \textit{Andromache}. Further, as noted above, it is important to examine tragedy as a series of events, and not simply as the static network of relationships that emerges at the end of the drama (when many of the characters are dead and/or destitute anyway).

I plan to continue this work between now and the conference in a few weeks, and I hope to offer substantial updates to the studies presented in this brief paper. In the meantime, however, I hope that these provocations towards computational approaches to messenger-type speeches prove useful beyond my own research, and that they demonstrate the need for discussion and debate about how to proceed most effectively.

\section{Appendix – Speakers in tragedy ranked by network centrality average}
\label{sec:appendix}

\begin{longtable}{|c|c|c|c|c|}
    \hline
    \textbf{ID} & \textbf{Rank} & \textbf{Title} & \textbf{Name} & \textbf{Centrality Average} \\
    \hline
    0 & 0 & Suppliant Women & choros & 1 \\ \hline
    10 & 1 & Seven Against Thebes & choros & 1 \\ \hline
    44 & 2 & Agamemnon & choros & 1 \\ \hline
    40 & 3 & Eumenides & choros & 1 \\ \hline
    29 & 4 & Libation Bearers & choros & 1 \\ \hline
    23 & 5 & Persians & choros & 1 \\ \hline
    197 & 6 & Oedipus Tyrannus & oidipous & 1.1 \\ \hline
    163 & 7 & Electra & elektra & 1.1 \\ \hline
    154 & 8 & Hecuba & hekabe & 1.1 \\ \hline
    218 & 9 & Ichneutae & choros & 1.1 \\ \hline
    221 & 10 & Ichneutae & kyllene & 1.1 \\ \hline
    224 & 11 & Electra & elektra & 1.1 \\ \hline
    71 & 12 & Rhesus & choros & 1.1 \\ \hline
    242 & 13 & Ajax & choros & 1.1 \\ \hline
    52 & 14 & The Trojan Women & hekabe & 1.1 \\ \hline
    18 & 15 & Prometheus Bound & prometheus & 1.2 \\ \hline
    192 & 16 & Philoctetes & neoptolemos & 1.2 \\ \hline
    231 & 17 & Antigone & choros & 1.3 \\ \hline
    206 & 18 & Oedipus at Colonus & oidipous & 1.3 \\ \hline
    115 & 19 & Iphigenia in Aulis & agamemnon & 1.4 \\ \hline
    62 & 20 & Suppliants & choros & 1.4 \\ \hline
    171 & 21 & Bacchae & choros & 1.5 \\ \hline
    182 & 22 & Trachiniae & choros & 1.5 \\ \hline
    28 & 23 & Libation Bearers & orestes & 1.5 \\ \hline
    155 & 24 & Hecuba & choros & 1.6 \\ \hline
    55 & 25 & The Trojan Women & choros & 1.6 \\ \hline
    128 & 26 & Ion & kreousa & 1.6 \\ \hline
    107 & 27 & Iphigenia in Tauris & iphigeneia & 1.7 \\ \hline
    134 & 28 & Heracles & amphitryon & 1.7 \\ \hline
    209 & 29 & Oedipus at Colonus & choros & 1.7 \\ \hline
    72 & 30 & Rhesus & hektor & 1.8 \\ \hline
    225 & 31 & Electra & choros & 1.8 \\ \hline
    143 & 32 & Helen & helene & 1.8 \\ \hline
    136 & 33 & Heracles & choros & 1.8 \\ \hline
    232 & 34 & Antigone & kreon & 1.8 \\ \hline
    170 & 35 & Bacchae & dionysos & 1.8 \\ \hline
    200 & 36 & Oedipus Tyrannus & choros & 1.8 \\ \hline
    127 & 37 & Ion & choros & 1.8 \\ \hline
    194 & 38 & Philoctetes & philoktetes & 1.8 \\ \hline
    216 & 39 & Ichneutae & silenos & 1.8 \\ \hline
    110 & 40 & Iphigenia in Tauris & choros & 1.9 \\ \hline
    64 & 41 & Suppliants & adrastos & 1.9 \\ \hline
    19 & 42 & Prometheus Bound & choros & 1.9 \\ \hline
    98 & 43 & Orestes & orestes & 1.9 \\ \hline
    45 & 44 & Agamemnon & klytaimestra & 2.1 \\ \hline
    241 & 45 & Ajax & aias & 2.2 \\ \hline
    24 & 46 & Persians & atossa & 2.2 \\ \hline
    145 & 47 & Helen & choros & 2.2 \\ \hline
    243 & 48 & Ajax & tekmessa & 2.2 \\ \hline
    165 & 49 & Electra & choros & 2.2 \\ \hline
    215 & 50 & Ichneutae & apollon & 2.2 \\ \hline
    97 & 51 & Orestes & choros & 2.3 \\ \hline
    126 & 52 & Ion & ion & 2.3 \\ \hline
    37 & 53 & Eumenides & apollon & 2.4 \\ \hline
    193 & 54 & Philoctetes & choros & 2.4 \\ \hline
    179 & 55 & Trachiniae & deianeira & 2.5 \\ \hline
    38 & 56 & Eumenides & orestes & 2.5 \\ \hline
    87 & 57 & Phoenissae & choros & 2.5 \\ \hline
    219 & 58 & Ichneutae & hemichoros a & 2.6 \\ \hline
    63 & 59 & Suppliants & theseus & 2.6 \\ \hline
    95 & 60 & Orestes & elektra & 2.6 \\ \hline
    207 & 61 & Oedipus at Colonus & antigone & 2.6 \\ \hline
    138 & 62 & Heracles & herakles & 2.6 \\ \hline
    90 & 63 & Phoenissae & kreon & 2.7 \\ \hline
    121 & 64 & Iphigenia in Aulis & klytaimestra & 2.7 \\ \hline
    223 & 65 & Electra & orestes & 2.7 \\ \hline
    174 & 66 & Bacchae & pentheus & 2.7 \\ \hline
    117 & 67 & Iphigenia in Aulis & choros & 2.7 \\ \hline
    164 & 68 & Electra & orestes & 2.7 \\ \hline
    21 & 69 & Prometheus Bound & io & 2.8 \\ \hline
    229 & 70 & Antigone & antigone & 2.8 \\ \hline
    86 & 71 & Phoenissae & antigone & 2.8 \\ \hline
    56 & 72 & The Trojan Women & talthybios & 2.9 \\ \hline
    220 & 73 & Ichneutae & hemichoros b & 3 \\ \hline
    48 & 74 & Agamemnon & kasandra & 3 \\ \hline
    181 & 75 & Trachiniae & hyllos & 3.1 \\ \hline
    108 & 76 & Iphigenia in Tauris & orestes & 3.1 \\ \hline
    160 & 77 & Hecuba & agamemnon & 3.1 \\ \hline
    25 & 78 & Persians & angelos & 3.2 \\ \hline
    32 & 79 & Libation Bearers & klytaimestra & 3.2 \\ \hline
    13 & 80 & Seven Against Thebes & antigone & 3.3 \\ \hline
    202 & 81 & Oedipus Tyrannus & iokaste & 3.3 \\ \hline
    146 & 82 & Helen & meneleos & 3.4 \\ \hline
    112 & 83 & Iphigenia in Tauris & thoas & 3.4 \\ \hline
    217 & 84 & Ichneutae & choros satyron & 3.4 \\ \hline
    26 & 85 & Persians & eidolon dareiou & 3.4 \\ \hline
    1 & 86 & Suppliant Women & danaos & 3.5 \\ \hline
    199 & 87 & Oedipus Tyrannus & kreon & 3.5 \\ \hline
    2 & 88 & Suppliant Women & basileus & 3.5 \\ \hline
    84 & 89 & Phoenissae & iokaste & 3.5 \\ \hline
    135 & 90 & Heracles & megara & 3.6 \\ \hline
    58 & 91 & The Trojan Women & andromache & 3.6 \\ \hline
    211 & 92 & Oedipus at Colonus & theseus & 3.6 \\ \hline
    47 & 93 & Agamemnon & agamemnon & 3.7 \\ \hline
    191 & 94 & Philoctetes & odysseus & 3.7 \\ \hline
    148 & 95 & Helen & angelos & 3.7 \\ \hline
    22 & 96 & Prometheus Bound & hermes & 3.7 \\ \hline
    6 & 97 & Suppliant Women & danais & 3.7 \\ \hline
    8 & 98 & Seven Against Thebes & eteokles & 3.7 \\ \hline
    5 & 99 & Suppliant Women & choros therapainon & 3.7 \\ \hline
    233 & 100 & Antigone & phylax & 3.8 \\ \hline
    99 & 101 & Orestes & menelaos & 3.8 \\ \hline
    150 & 102 & Helen & theoklymenos & 3.9 \\ \hline
    101 & 103 & Orestes & pylades & 3.9 \\ \hline
    80 & 104 & Rhesus & athena & 3.9 \\ \hline
    122 & 105 & Iphigenia in Aulis & iphigeneia & 4 \\ \hline
    161 & 106 & Hecuba & polymestor & 4 \\ \hline
    173 & 107 & Bacchae & kadmos & 4.2 \\ \hline
    78 & 108 & Rhesus & odysseus & 4.2 \\ \hline
    162 & 109 & Electra & autourgos & 4.2 \\ \hline
    222 & 110 & Electra & paidagogos & 4.2 \\ \hline
    20 & 111 & Prometheus Bound & okeanos & 4.3 \\ \hline
    31 & 112 & Libation Bearers & oiketes & 4.3 \\ \hline
    41 & 113 & Eumenides & athena & 4.3 \\ \hline
    12 & 114 & Seven Against Thebes & hemichorion b & 4.3 \\ \hline
    4 & 115 & Suppliant Women & choros danaidon & 4.4 \\ \hline
    39 & 116 & Eumenides & klytaimestras eidolon & 4.4 \\ \hline
    16 & 117 & Prometheus Bound & kratos & 4.4 \\ \hline
    156 & 118 & Hecuba & polyxene & 4.4 \\ \hline
    131 & 119 & Ion & therapon & 4.4 \\ \hline
    240 & 120 & Ajax & odysseus & 4.4 \\ \hline
    166 & 121 & Electra & presbys & 4.5 \\ \hline
    168 & 122 & Electra & klytaimestra & 4.5 \\ \hline
    227 & 123 & Electra & klytaimnestra & 4.5 \\ \hline
    109 & 124 & Iphigenia in Tauris & pylades & 4.5 \\ \hline
    27 & 125 & Persians & xerxes & 4.5 \\ \hline
    46 & 126 & Agamemnon & keryx & 4.5 \\ \hline
    9 & 127 & Seven Against Thebes & angelos & 4.5 \\ \hline
    7 & 128 & Suppliant Women & therapaina & 4.5 \\ \hline
    230 & 129 & Antigone & ismene & 4.5 \\ \hline
    247 & 130 & Ajax & teukros & 4.6 \\ \hline
    11 & 131 & Seven Against Thebes & hemichorion a & 4.6 \\ \hline
    3 & 132 & Suppliant Women & keryx & 4.6 \\ \hline
    65 & 133 & Suppliants & keryx & 4.7 \\ \hline
    36 & 134 & Eumenides & pythias & 4.7 \\ \hline
    57 & 135 & The Trojan Women & kasandra & 4.7 \\ \hline
    61 & 136 & Suppliants & aithra & 4.7 \\ \hline
    123 & 137 & Iphigenia in Aulis & achilleus & 4.7 \\ \hline
    196 & 138 & Philoctetes & herakles & 4.8 \\ \hline
    79 & 139 & Rhesus & diomedes & 4.9 \\ \hline
    17 & 140 & Prometheus Bound & hephaistos & 4.9 \\ \hline
    203 & 141 & Oedipus Tyrannus & angelos & 5 \\ \hline
    212 & 142 & Oedipus at Colonus & kreon & 5 \\ \hline
    113 & 143 & Iphigenia in Tauris & angelos & 5 \\ \hline
    142 & 144 & Heracles & theseus & 5.1 \\ \hline
    49 & 145 & Agamemnon & aigisthos & 5.1 \\ \hline
    132 & 146 & Ion & prophetis & 5.1 \\ \hline
    176 & 147 & Bacchae & angelos & 5.2 \\ \hline
    82 & 148 & Rhesus & heniochos & 5.2 \\ \hline
    190 & 149 & Trachiniae & herakles & 5.2 \\ \hline
    195 & 150 & Philoctetes & emporos & 5.2 \\ \hline
    157 & 151 & Hecuba & odysseus & 5.3 \\ \hline
    137 & 152 & Heracles & lykos & 5.3 \\ \hline
    59 & 153 & The Trojan Women & menelaos & 5.3 \\ \hline
    60 & 154 & The Trojan Women & helene & 5.3 \\ \hline
    133 & 155 & Ion & athena & 5.4 \\ \hline
    34 & 156 & Libation Bearers & aigisthos & 5.4 \\ \hline
    184 & 157 & Trachiniae & lichas & 5.5 \\ \hline
    89 & 158 & Phoenissae & eteokles & 5.5 \\ \hline
    116 & 159 & Iphigenia in Aulis & presbytes & 5.5 \\ \hline
    15 & 160 & Seven Against Thebes & keryx & 5.6 \\ \hline
    129 & 161 & Ion & xouthos & 5.6 \\ \hline
    93 & 162 & Phoenissae & angelos & 5.6 \\ \hline
    188 & 163 & Trachiniae & trophos & 5.6 \\ \hline
    30 & 164 & Libation Bearers & elektra & 5.6 \\ \hline
    226 & 165 & Electra & chrysothemis & 5.7 \\ \hline
    183 & 166 & Trachiniae & angelos & 5.7 \\ \hline
    210 & 167 & Oedipus at Colonus & ismene & 5.7 \\ \hline
    234 & 168 & Antigone & haimon & 5.7 \\ \hline
    159 & 169 & Hecuba & therapaina & 5.7 \\ \hline
    235 & 170 & Antigone & teiresias & 5.8 \\ \hline
    169 & 171 & Electra & dioskoyroi & 5.8 \\ \hline
    70 & 172 & Suppliants & athena & 5.8 \\ \hline
    213 & 173 & Oedipus at Colonus & polyneikes & 5.8 \\ \hline
    66 & 174 & Suppliants & angelos & 5.8 \\ \hline
    178 & 175 & Bacchae & agaue & 5.9 \\ \hline
    74 & 176 & Rhesus & dolon & 5.9 \\ \hline
    172 & 177 & Bacchae & teiresias & 5.9 \\ \hline
    81 & 178 & Rhesus & alexandros & 6 \\ \hline
    147 & 179 & Helen & graus & 6 \\ \hline
    35 & 180 & Libation Bearers & pylades & 6.1 \\ \hline
    42 & 181 & Eumenides & propompoi & 6.1 \\ \hline
    114 & 182 & Iphigenia in Tauris & athena & 6.1 \\ \hline
    201 & 183 & Oedipus Tyrannus & teiresias & 6.2 \\ \hline
    53 & 184 & The Trojan Women & hemichorion a & 6.2 \\ \hline
    33 & 185 & Libation Bearers & trophos & 6.3 \\ \hline
    228 & 186 & Electra & aigisthos & 6.3 \\ \hline
    244 & 187 & Ajax & angelos & 6.3 \\ \hline
    68 & 188 & Suppliants & iphis & 6.4 \\ \hline
    43 & 189 & Agamemnon & phylax & 6.4 \\ \hline
    94 & 190 & Phoenissae & oidipous & 6.4 \\ \hline
    149 & 191 & Helen & theonoe & 6.4 \\ \hline
    236 & 192 & Antigone & angelos & 6.5 \\ \hline
    124 & 193 & Iphigenia in Aulis & angelos & 6.5 \\ \hline
    130 & 194 & Ion & presbytes & 6.5 \\ \hline
    96 & 195 & Orestes & helene & 6.5 \\ \hline
    14 & 196 & Seven Against Thebes & ismene & 6.5 \\ \hline
    205 & 197 & Oedipus Tyrannus & exangelos & 6.6 \\ \hline
    76 & 198 & Rhesus & angelos & 6.6 \\ \hline
    54 & 199 & The Trojan Women & hemichorion b & 6.7 \\ \hline
    83 & 200 & Rhesus & mousa & 6.7 \\ \hline
    204 & 201 & Oedipus Tyrannus & therapon & 6.7 \\ \hline
    187 & 202 & Trachiniae & hemichorion & 6.7 \\ \hline
    185 & 203 & Trachiniae & hemichorion 1 & 6.7 \\ \hline
    245 & 204 & Ajax & hemichorion 1 & 6.7 \\ \hline
    100 & 205 & Orestes & tyndareos & 6.8 \\ \hline
    140 & 206 & Heracles & lyssa & 6.8 \\ \hline
    238 & 207 & Antigone & exangelos & 6.9 \\ \hline
    111 & 208 & Iphigenia in Tauris & boukolos & 6.9 \\ \hline
    67 & 209 & Suppliants & euadne & 6.9 \\ \hline
    73 & 210 & Rhesus & aineias & 7 \\ \hline
    158 & 211 & Hecuba & talthybios & 7 \\ \hline
    167 & 212 & Electra & angelos & 7.1 \\ \hline
    246 & 213 & Ajax & hemichorion 2 & 7.1 \\ \hline
    139 & 214 & Heracles & iris & 7.2 \\ \hline
    91 & 215 & Phoenissae & teiresias & 7.3 \\ \hline
    118 & 216 & Iphigenia in Aulis & menelaos & 7.4 \\ \hline
    208 & 217 & Oedipus at Colonus & xenos & 7.4 \\ \hline
    175 & 218 & Bacchae & therapon & 7.4 \\ \hline
    198 & 219 & Oedipus Tyrannus & hiereus & 7.4 \\ \hline
    186 & 220 & Trachiniae & hemichorion 2 & 7.4 \\ \hline
    104 & 221 & Orestes & ermione & 7.4 \\ \hline
    88 & 222 & Phoenissae & polyneikes & 7.5 \\ \hline
    239 & 223 & Ajax & athena & 7.6 \\ \hline
    237 & 224 & Antigone & eurydike & 7.6 \\ \hline
    141 & 225 & Heracles & angelos & 7.7 \\ \hline
    151 & 226 & Helen & therapon & 7.7 \\ \hline
    152 & 227 & Helen & dioskoroi & 7.7 \\ \hline
    248 & 228 & Ajax & menelaos & 7.8 \\ \hline
    125 & 229 & Ion & hermes & 7.8 \\ \hline
    50 & 230 & The Trojan Women & poseidon & 7.9 \\ \hline
    119 & 231 & Iphigenia in Aulis & angelos a & 7.9 \\ \hline
    177 & 232 & Bacchae & angelos b & 7.9 \\ \hline
    214 & 233 & Oedipus at Colonus & angelos & 8 \\ \hline
    153 & 234 & Hecuba & polydorou eidolon & 8 \\ \hline
    120 & 235 & Iphigenia in Aulis & xorosandronargeion & 8.2 \\ \hline
    75 & 236 & Rhesus & angelos poimen & 8.3 \\ \hline
    51 & 237 & The Trojan Women & athena & 8.3 \\ \hline
    249 & 238 & Ajax & agamemnon & 8.4 \\ \hline
    144 & 239 & Helen & teukros & 8.4 \\ \hline
    106 & 240 & Orestes & apollon & 8.5 \\ \hline
    92 & 241 & Phoenissae & menoikeus & 8.6 \\ \hline
    105 & 242 & Orestes & phryx & 8.6 \\ \hline
    69 & 243 & Suppliants & paides & 8.7 \\ \hline
    189 & 244 & Trachiniae & presbys & 9.1 \\ \hline
    102 & 245 & Orestes & angelos & 9.4 \\ \hline
    180 & 246 & Trachiniae & therapaina & 9.4 \\ \hline
    77 & 247 & Rhesus & resos & 9.6 \\ \hline
    85 & 248 & Phoenissae & paidagogos & 9.7 \\ \hline
    103 & 249 & Orestes & hemichoros & 9.7 \\ \hline
\end{longtable}


\section{Declaration on Generative AI}

The author has not employed any Generative AI tools.

\bibliography{zotero-refs}

\end{document}
